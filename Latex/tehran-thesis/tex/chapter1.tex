\chapter{مقدمه}
\subsection{مقدمه ای بر \lr{5G} }
\lr{5G}،
مخابرات نسل پنجم
سیستم های بیسیم \LTRfootnote{Wireless} وشبکه های مخابراتی بعد از نسل چهارم می باشد که تکاملی از لایه ی فیزیکی در تکنولوژی شبکه های مخابراتی سیار همانند \lr{LTE} می باشد که نسبت به \lr{4G} سرعت و پوشش بهتری را فراهم می کند.  
\lr{5G}
نوع جدیدی از شبکه را ایجاد می کند که به منظور اتصال تقریبا همه و همه چیز با هم از جمله ماشین ها، اشیاء و دستگاه ها ساخته شده است.
\lr{5G}
 فناوری بی سیم برای ارائه سرعت داده های چند گیگابیت بر ثانیه ، تأخیر فوق العاده کم ، قابلیت اطمینان بیشتر ، ظرفیت شبکه گسترده، افزایش در دسترس بودن و تجربه کاربری یکنواخت تر به کاربران بیشتر است. عملکرد بالاتر و بهره وری بهبود یافته باعث افزایش تجربیات کاربر جدید شده و صنایع جدیدی را به هم متصل می کند.
 
 
تکنولوژی سیگنال  \lr{5G} برای پوشش فراگیرتر و بازدهی بهتر سیگنال ایجاد شده است. این پیشرفت ها منجر به تغییراتی از قبیل\lr{IOT} \LTRfootnote{Internet of Things} و \lr{Pervasive Computing} در آینده ی نزدیک خواهد شد.
همچنین \lr{5G} منجر به توسعه و بهبود سرویس های مخابراتی و اینترنتی سیار و در ورای آن، ایجاد تجربه ی بهتری برای مصرف کنندگان خواهد شد.\newline
برای توسعه ی اینترنت سیار و \lr{IOT}، نیاز داریم تا شبکه های\lr{5G}، معنای اولیه برای دسترسی شبکه برای ارتباط انسان ها با یکدیگر و ارتباط ماشین با انسان گردد.
به طور کلی، 
\lr{5G}
 در سه نوع سرویس اصلی متصل از جمله پهن باند تلفن همراه ، ارتباطات مهم برای ماموریت و IoT عظیم استفاده می شود.
 
 از آنجا که
\lr{5G}
 تکامل می یابد و کمتر به زیرساخت های \lr{4G} وابسته می شود و طیف بیشتری در دسترس قرار می گیرد،
 تخمین ها سرعت بارگیری را حداکثر 1000 بار سریعتر از \lr{4G} قرار می دهد، که بالقوه از 
 \lr{10Gbps} 
بیشتر است، که به شما امکان می دهد تا در کمتر از ثانیه فیلم کامل
\lr{HD}
   را بارگیری کنید.
   برخی تخمین ها محافظه کارتر هستند ، اما حتی محافظه کار ترین آن را چندین ده برابر سریعتر از \lr{4G} قرار می دهد.
   \begin{table}[H]
 \caption {سرعت دانلود در نسل های مختلف مخابراتی} \label{tab:title3} 
 \begin{center}
  \begin{tabular}{||c c ||} \hline
  نوع شبکه 
 \\ [0.5ex]  
  \hline\hline
  Number of cluster S & 3 \\ 
  \hline
  Noise power density & -174dBm/Hz\\
  \hline
  Bandwidth & 120KHz \\
  \hline
 Maxmimun transmit Power & 10dBm \\
  \hline
  Circuit Power of whole RRHs & 10dBm \\
  \hline
  Variance of quantization noise & $10^{-2}$ \\
  \hline
   Maxmimun fronthaul link's rate & 20bits/sec/Hz \\
  \hline
  Minimum data rate &  1bits/sec/Hz \\ [1ex] 
  \hline
 \end{tabular}
 \end{center}
 \end{table}
\subsection{تاریخچه مخابرات}
در ابتدا می خواهیم بدانیم که چه چیزی منجر به رفتن محققان به سوی  \lr{5G} شده است. یکی از دلایل مهم، سرعت و نرخ انتقال بیشتری است که در ادامه به آن می پردازیم.
در ابتدا نیاز بشریت به ارتباط تلفنی (انتقال بدون سیم به صورت زمان حقیقی \LTRfootnote{Real Time} بشریت را به سمت نسل اول ارتباطات \lr{1G} سوق داده است . نسل دوم ارتباطات \lr{2G} با سرویس های انتقال پیام کوتاه ایجاد شد. همچنین با موفقیت تکنولوژی شبکه های منطقه ای بیسیم، اتصال به داده های اینترنتی مورد توجه عموم مردم قرار گرفت که پلی به سوی نسل سوم ارتباطات \lr{3G} را فراهم نمود. به طور منطقی پله ی بعدی گام برداشتن در راستای کوچک شدن لپ تاپ و در آمیختن آن با تلفن که امروزه به صورت تلفن هوشمند\LTRfootnote{smart phone} است و دسترسی به  اینترنت، پهنای باند بالا و داده ها در نقاط مختلف جهان بوده است که \lr{4G} یا نسل چهارم را به همراه داشته است.
با توجه به افزایش تعداد کاربران تلفن های
هوشمند و تبلت ها و افزایش نرخ ارسال اطلاعات و داده ها در طی سال
های اخیر طبق پیش بینی های سیسکو میزان ترافیک \lr{IP} طی سالهای اخیر
  چندین برابر افزایش خواهد یافت.
در نتیجه اپراتورها برای حل این مشکل و خدمات
دهی بهتر ناچار به افزایش ظرفیت شبکه می باشند. با توجه
به این که نرخ داده و ظرفیت در سیستم های نسل چهارم به ظرفیت
شانون نزدیک شده است، در نتیجه روش هایی که برای
افزایش ظرفیت شبکه مورد استفاده می گیرند که به شرح زیر است:
\begin{itemize}
\item
استفاده از تکنیک \lr{Massive Mimo}
\item
استفاده از روش های پردازش های ابری
\item
\lr{Software Defined Networking}
\item
\lr{Extreme Densification}
\item
روش \lr{mm Wave}
\end{itemize}
\section{مقدمه ای بر ساختار \lr{ORAN}}
\lr{Open RAN}(\lr{ORAN})
تبسیط و ترکیبی از دو ساختار \lr{C-RAN} و \lr{xRAN} می باشد که انتظار می رود که در فناوری نسل پنجم مخابرات مورد استفاده قرار گرفته و منجر به بهبود عملکرد شبکه های دسترسی رادیویی \lr{RAN} گردد.
\subsection{\lr{C-RAN}} 
ایده اصلی \lr{C-RAN} جداسازی بخش رادیویی (\lr{RRH}) 
\LTRfootnote{Radio Remote Head}
 از واحد پردازشی باند پایه (\lr{BBU})
 \LTRfootnote{Baseband Unit}
  است.
از تجمیع \lr{BBU} ها بر روی سرور ابری، \lr{BBU-Pool} ایجاد می شود.
در این ساختار، در راستای بهینه سازی عملکرد \lr{BBU}
 ها در مواجهه باایستگاههای پایه پر ترافیک و کم ترافیک،
 \lr{BBU}ها به صورت یک مجموعه ی واحد تحت عنوان 
\lr{BBU Pool}
 در آمده اند که این مجموعه بین چندین سلول 
 به اشتراک گزارده شده و مطابق شکل زیر مجازی سازی
می شود. 
در توضیح بیشتر این ساختار می توان این گونه
عنوان کرد که \lr{BBU Pool} به عنوان یک خوشه ی مجازی
در نظر گرفته می شود که شامل پردازش گرهایی می باشد
که پردازش های باند پایه را انجام می دهند. ارتباط بین
  \lr{BBU}ها در ساختار های فعلی به شکل  $X_2$ برقرار می شود
که در این ساختار ارتباط بین خوشه ها از فرم جدید $X_2$
تحت عنوان  $X_2 +$برقرار می شود.
\newline
در شکل \ref{fig:C-RAN} ساختار کلی شبکه ی  \lr{C-RAN} در سیستم های
\lr{ LTE}
 نمایش داده شده است. همان طور که در شکل قابل
مشاهده می باشد ساختار کلی شبکه  \lr{C-RAN} به دو بخش
 \lr{backhaul} و \lr{fronthaul} تقسیم بندی شده است. بخش
 \lr{fronthaul}شبکه به مرحله ی اتصال سایت های \lr{ RRH}به
 به \lr{BBU Pool} به اتصال \lr{backhaul} و بخش \lr{BBU Pool}
هسته ی شبکه ی سیار اطلاق می شود. همان گونه که قبلا
ذکر شد  \lr{ RRH}ها در نزدیکی انتن نصب شده و از طریق
لینک های انتقالی نوری با پهنای باند وسیع و تاخیر کم به
پردازشگرهای قوی در  \lr{BBU}متصل می شوند. توسط این
لینک های انتقالی است که سیگنال های دیجیتالی باند
پایه از نوع \lr{IQ} بین \lr{RRH} و \lr{BBU} انتقال می یابند \cite{checko2015cloud}.
\begin{figure}[H]
  \centering
    \includegraphics[width=\textwidth]{./fig/CRAN}
  \caption{ساختار شبکه ی \lr{C-RAN} \cite{checko2015cloud}}
  \label{fig:C-RAN}
\end{figure}
\subsection{\lr{xRAN}}
\lr{xRAN}
در سال ۲۰۱۶ با هدف استانداردسازی یک جایگزین انعطاف پذیر و باز برای \lr{RAN}
مبتنی بر سخت افزار سنتی بدست آمده است.
 در این ساختار، سه حوزه ی مهم مورد بررسی قرار گرفته است.
اولین حوزه ی مورد بررسی، جداسازی بخش \lr{control plane} از 
\lr{user plane}
می باشد. حوزه ی دوم،
ساختن یک پشته نرم افزاری eNodeB مدولار که از سخت افزار COTS استفاده می کند، می باشد.
حوزه ی سوم مورد بررسی، انتشار رابط های باز شمال و جنوب است
\cite{xran}.
\subsection{\lr{ORAN}}
\lr{ORAN}، المانهای شبکه ی دسترسی رادیویی را مجازی می کند ، آنها را جدا کرده و رابط های باز مناسب را 
برای اتصال این عناصر
تعیین می کند. همچنین، 
\lr{ORAN}
از روشهای یادگیری ماشین برای هوشمندسازی لایه های 
\lr{RAN}
 استفاده می نماید. 
 در ساختار نوآورانه ی 
 \lr{ORAN}
 نرم افزار قابل برنامه ریزی 
 \lr{RAN}
 از سخت افزار جدا می شود.
  یکی از مهم ترین خصوصیات
  \lr{ORAN}
  رابط کاربری باز است که به اپراتورهای موبایل این قابلیت را می دهد تا بتوانند سرویس های مورد نیاز خود را تعریف نمایند.
  مفهوم 
  \lr{SDN}
  \LTRfootnote{software defined network}
  که مبنی بر جداسازی 
   بخش \lr{control plane} از 
\lr{user plane}
می باشد، در ساختار 
\lr{ORAN}
مورد بررسی قرار می گیرد.
این جداسازی منجر به بهبود 
\lr{RRM}
برای استفاده از زمان غیر واقعی و زمان نزدیک به واقعی در کنترلگر هوشمند شبکه ی دسترسی رادیویی می گردد.
در ساختار
\lr{ORAN}،
واحد توزیع شده \lr{DU}،
نود منطقی می باشد که شامل لایه های 
\lr{RLC}
،
\lr{MAC}،
و
\lr{High-PHY}
است.
علاوه بر این، واحد مرکزی 
\lr{CU}
نود منطقی است که شامل لایه های 
\lr{RRC}،
\lr{SDAP} 
و 
\lr{PDCP}
می باشد.
نود منطقی واحد رادیویی
\lr{RU}
نیز، شامل لایه ی 
\lr{LOW-PHY}
و بخش پردازش رادیویی می باشد.
\lr{ORAN}
،
رابطهایی از جمله رابط 
\lr{fronthaul}
باز را شامل می شود که بخش \lr{DU} را به \lr{RU} متصل می نماید
(رابط 
\lr{E2}). 
همچنین
 رابط \lr{A1}
 بین لایه ی 
  \lr{orchestration/NMS}
  که شامل 
  تابع غیر واقعی زمان است و 
  \lr{eNB/qNB}
  که شامل تابع نزدیک به زمان است. 
\section{مجازی سازی توابع شبکه}
برای بهبود سرویس دهی در نسل پنجم مخابرات، جداسازی المان های نرم افزاری و سخت افزاری شبکه صورت گرفته است و به عنوان 
مجازی سازی توابع شبکه (\lr{NFV}) \LTRfootnote{network function virtualization}
معرفی شده است.
  حال توابع شبکه ی مجازی
  \lr{VNF}
  \LTRfootnote{Virtual network function} ،
  بلوکهای توابع سیستم هستند.
در نسل پنجم مخابرات 
  انتظار می رود که
   میزبان چندین سرویس
   با نیازهای مختلف به طور همزمان
    باشند.
 \section{برش شبکه}
 برش شبکه
 \LTRfootnote{Network Slicing}
به عنوان راه حلی برای چنین تقاضا در نظر گرفته شده است.
یک برش شبکه، یک شبکه منطقی \lr{end-to-end} است که خدمات  با نیازهای خاص را ارائه می دهد.
 چندین برش شبکه
در یک زیرساخت یکسان
  اجرا و مدیریت می شوند و
به طور مستقل کار کنید.
پیاده سازی های مختلفی از برش شبکه وجود دارد که شامل برش هسته ی شبکه، برش \lr{RAN} و برش هر دو بخش می باشد.
در برش هسته ی شبکه، برش تنها در بخش هسته ی شبکه است و تمام واسط ها و فرآیندها، بدون تغییر باقی می مانند.
در برش \lr{RAN}، 
برش های \lr{RAN}، بر روی   